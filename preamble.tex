\usepackage[T1]{fontenc}
% \usepackage{mlmodern}
% many useful symbols
\usepackage{textcomp}
\usepackage[english]{babel}
\usepackage{amsmath, amsthm, amssymb}
\usepackage{wasysym}
% for \lightning
\usepackage{stmaryrd}
\usepackage{color}
\usepackage{geometry}
\usepackage{tikz-cd}
\usepackage{tikz}
\usepackage{bold-extra}
% for \coloneqq
\usepackage{mathtools}
% References
\usepackage{hyperref}
\usepackage[capitalise]{cleveref}

% Bibliography
\usepackage[backend=biber, style=alphabetic]{biblatex}
\addbibresource{bibliography.bib}

% remove citations
% \renewcommand{\cite}[1]{#1}
% \renewcommand{\nocite}[1]{}
% \def\printbibliography{}

\hypersetup{
    colorlinks = true, % links instead of boxes
    urlcolor   = cyan, % external hyperlinks
    linkcolor  = blue, % internal links
    citecolor  = cyan   % citations
}

% Highlight missing references in red
\makeatletter
\def\@setref#1#2#3{% csname, extract group, refname
    \ifx#1\relax
        \protect\G@refundefinedtrue
        \nfss@text{\reset@font\textcolor{red}{#3}}%
        \@latex@warning{%
            Reference `#3' on page \thepage \space undefined%
        }%
    \else
        \expandafter\Hy@setref@link#1\@empty\@empty\@empty\@nil{#2}%
    \fi
}
\makeatother

% % Have blank lines between paragraphs
% \nonzeroparskip
% % Remove indentation globally
% \setlength{\parindent}{0pt}

% Definitions
\def\R{\mathbb{R}}
\def\C{\mathbb{C}}
\def\Q{\mathbb{Q}}
\def\N{\mathbb{N}}
\def\Z{\mathbb{Z}}
\def\K{K}
\def\k{\kappa}
\def\Ks{\overline{\K}}
\def\Kx{\Ks^\times}
\def\ks{\overline{\kappa}}
\def\Knr{\K_{\text{nr}}}
\def\Qp{\mathbb{Q}_p}
\def\Zp{\Z_p}
\def\Zn{\Z/n\Z}
\def\Gmod{\mathsf{Mod}_\mathsf{G}}
\def\Hmod{\mathsf{Mod}_\mathsf{H}}
\def\GHmod{\mathsf{Mod}_{\mathsf{G}/\mathsf{H}}}
\def\Gfmod{\mathsf{Mod}_{\mathsf{G}}^f}
\def\Ab{\mathsf{Ab}}

% Operators
\newcommand{\Gal}[1]{\operatorname{Gal}\left( \overline{#1}/#1 \right)}
\newcommand{\Aut}[1]{\operatorname{Aut}\left( #1 \right)}
\renewcommand{\H}[3]{H^{#1}( #2, \, #3 )}
\newcommand{\HH}[2]{H^2(#1, \, #2 )}
\newcommand{\Ind}[2]{\operatorname{I}_{#1}(#2)}
\newcommand{\ZG}[1]{\Z[#1]}
\newcommand{\Br}[1]{\operatorname{Br}{(#1)}}
\newcommand{\cd}[1]{\operatorname{cd}{(#1)}}
\newcommand{\cdp}[1]{\operatorname{cd}_p{(#1)}}
\newcommand{\cdl}[1]{\operatorname{cd}_l{(#1)}}
\DeclareMathOperator{\St}{St}
\DeclareMathOperator{\Hom}{Hom}
\DeclareMathOperator{\Ext}{Ext}
\DeclareMathOperator{\Res}{Res}
\DeclareMathOperator{\Cor}{Cor}
\DeclareMathOperator{\Inf}{Inf}
\DeclareMathOperator{\sh}{Sh}
\DeclareMathOperator{\coker}{Coker}
\DeclareMathOperator{\id}{id}

% Arrow related
\newcommand\Iso{\xrightarrow{
        \,\smash{\raisebox{-0.65ex}{\ensuremath{\scriptstyle\sim}}}\,}}
\newcommand\restrict[1]{\raisebox{-.5ex}{$|$}_{#1}}

% Display math
\newcommand{\ssfrac}[2]{
    \raisebox{+0.3ex}{$#1$}
    /
    \raisebox{-0.3ex}{$#2$}
}
% Inline math
\newcommand{\sfrac}[2]{
    \raisebox{+0.3ex}{\scalebox{0.9}{$#1$}}
    /
    \raisebox{-0.3ex}{\scalebox{0.9}{$#2$}}
}

% use bullets for items
\renewcommand{\labelitemii}{$\circ$}
\renewcommand{\Im}{\operatorname{im}}

% Number a specific equation within an unnumbered align
\newcommand\numberthis{\addtocounter{equation}{1}\tag{\theequation}}

% Spectral Sequences
\usetikzlibrary{matrix}
% No white line in equal arrows
\usetikzlibrary{decorations.markings}
\tikzset{double line with arrow/.style args={#1,#2}{decorate,decoration={markings,%
                    mark=at position 0 with {\coordinate (ta-base-1) at (0,1pt);
                            \coordinate (ta-base-2) at (0,-1pt);},
                    mark=at position 1 with {\draw[#1] (ta-base-1) -- (0,1pt);
                            \draw[#2] (ta-base-2) -- (0,-1pt);
                        }}}}
\tikzset{Equal/.style={-,double line with arrow={-,-}}}

\theoremstyle{plain}
\newtheorem{theorem}{Theorem}[section]
\newtheorem{lemma}[theorem]{Lemma}
\newtheorem{proposition}[theorem]{Proposition}
\newtheorem{corollary}[theorem]{Corollary}

\theoremstyle{definition}
\newtheorem{definition}[theorem]{Definition}
\newtheorem{example}[theorem]{Example}

\theoremstyle{remark}
\newtheorem{remark}[theorem]{Remark}

\chapterstyle{madsen}