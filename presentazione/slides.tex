\documentclass[pdf]{beamer}

\usepackage[T1]{fontenc}
\usepackage{textcomp}
\usepackage[italian]{babel}
\usepackage{hyperref}
\usepackage{amsmath, amssymb, amsthm, amsfonts}
\usepackage{geometry}
\usepackage{empheq}
\usepackage{tikz-cd}

\hypersetup{
    colorlinks = true, % links instead of boxes
    urlcolor   = blue, % external hyperlinks
    linkcolor  = white, % internal links
    citecolor  = red   % citations
}

% Definitions
\def\R{\mathbb{R}}
\def\C{\mathbb{C}}
\def\Q{\mathbb{Q}}
\def\N{\mathbb{N}}
\def\Z{\mathbb{Z}}
\def\K{K}
\def\k{\kappa}
\def\Ks{\overline{\K}}
\def\Kx{\Ks^\times}
\def\ks{\overline{\kappa}}
\def\Knr{\K_{\text{nr}}}
\def\Qp{\mathbb{Q}_p}
\def\Zp{\Z_p}
\def\Zn{\Z/n\Z}
\def\Gmod{\mathsf{Mod}_\mathsf{G}}
\def\Hmod{\mathsf{Mod}_\mathsf{H}}
\def\GHmod{\mathsf{Mod}_{\mathsf{G}/\mathsf{H}}}
\def\Gfmod{\mathsf{Mod}_{\mathsf{G}}^f}
\def\Ab{\mathsf{Ab}}

% Operators
\newcommand{\Gal}[1]{\operatorname{Gal}\left( \overline{#1}/#1 \right)}
\newcommand{\Aut}[1]{\operatorname{Aut}\left( #1 \right)}
\renewcommand{\H}[3]{H^{#1}( #2, \, #3 )}
\newcommand{\HH}[2]{H^2(#1, \, #2 )}
\newcommand{\Ind}[2]{\operatorname{I}_{#1}(#2)}
\newcommand{\ZG}[1]{\Z[#1]}
\newcommand{\Br}[1]{\operatorname{Br}{(#1)}}
\newcommand{\cd}[1]{\operatorname{cd}{(#1)}}
\newcommand{\cdp}[1]{\operatorname{cd}_p{(#1)}}
\newcommand{\cdl}[1]{\operatorname{cd}_l{(#1)}}
\DeclareMathOperator{\St}{St}
\DeclareMathOperator{\Hom}{Hom}
\DeclareMathOperator{\Ext}{Ext}
\DeclareMathOperator{\Res}{Res}
\DeclareMathOperator{\Cor}{Cor}
\DeclareMathOperator{\Inf}{Inf}
\DeclareMathOperator{\sh}{Sh}
\DeclareMathOperator{\coker}{Coker}
\DeclareMathOperator{\id}{id}

\renewcommand{\Im}{\operatorname{Im}}

\newcommand\numberthis{\addtocounter{equation}{1}\tag{\theequation}}

% Translation of environments
\deftranslation[to=italian]{Example}{Esempio}
\deftranslation[to=italian]{Definition}{Definizione}
\deftranslation[to=italian]{Theorem}{Teorema}

% usage: \boxedeq{description}{equation}
\newcommand{\boxedeq}[2]{\begin{empheq}[box={\fboxsep=6pt\fbox}]{align}\label{#1}#2\end{empheq}}

\usetheme{Madrid}
\useinnertheme{rectangles}
\useoutertheme{split}

\title{Dualità Locale di Tate}
\date{14 giugno 2024}
\author{Francesco Minnocci}

\begin{document}
\begin{frame}
	\titlepage
\end{frame}

% Cosa vogliamo fare?
\begin{frame}[fragile]{Prologo}
	\begin{itemize}
		\item<1-> \textit{Class field theory}: descrivere le estensioni abeliane di un campo $K$.
		      % ovvero le estensioni di Galois con gruppo abeliano, termini delle proprietà aritmetiche di K.

		\item<2-> $K/\Q \rightsquigarrow K/\Q_p$\only<3->{, oppure $K/\mathbb{F}_q(t) \rightsquigarrow K/\mathbb{F}_q((t))$.}
		      %  Ad esempio possiamo prendere K estensione finita di \Q (campo globale) o ad un suo completamento metrico rispetto ad un valore assoluto non archimedeo (campo locale).

		\item<4-> Sia $\Ks$ una chiusura separabile di $K$, e $K^\text{ab}$ la massima estensione abeliana di $K$.
		      % cioè il composto delle estensioni abeliane finite contenute in \Ks
		      % Per corrispondenza di Galois, vogliamo quindi studiare il gruppo di Galois di K^ab su K
		      \only<5->{\begin{align*}
				      \operatorname{Gal}(K^\text{ab}/K) & \only<6->{=\operatorname{Gal}(\Ks/K)^\text{ab}}                              \\
				      \only<7->{                        & =\varprojlim_{\mathclap{L/K\text{ abeliana finita}}}\operatorname{Gal}(L/K)}
			      \end{align*}}
		      % Infatti, questo è un gruppo profinito (in quanto limite inverso dei gruppi di Galois delle estensioni abeliani finite) e i suoi sottogruppi chiusi corrispondono alle estensioni abeliane di K.
		\item<8-> Un gruppo topologico $G$ è profinito $\iff$ è compatto, Hausdorff e totalmente sconnesso.
		      % In particolare, i sottogruppi aperti coincidono con quelli chiusi di indice finito. In tal caso, possiamo scrivere
		      \[
			      G=\varprojlim_{\mathclap{N\lhd G\text{ aperto}}}G/N
		      \]
	\end{itemize}
	\begin{definition}<9->
		Dato $G$ gruppo discreto, $\widehat{G}\coloneqq\varprojlim_N\nolimits{G/N}$ è il completamento profinito di $G$.
	\end{definition}
\end{frame}

\begin{frame}[fragile]{Local Class Field Theory}
	% Ora ci concentriamo sui campi locali:
	\begin{definition}[Campo locale]<1->
		Un campo completo rispetto ad una valutazione discreta di rango 1 con campo residuo finito.
	\end{definition}
	\begin{example}<2->
		Un campo locale è un estensione finita di $\Q_p$ oppure di $\mathbb{F}_q((t))$.
	\end{example}
	% a seconda della sua caratteristica 0 o p primo. Per i campi p-adici, mostreremo che c'è un isomorfismo
	\begin{theorem}<3->
		Sia $K$ un campo $p$-adico con gruppo di Galois assoluto $\Gamma=\operatorname{Gal}{(\Ks /K)}$. Allora,
		\[
			\Gamma^\text{ab} \simeq \widehat{K^\times}
		\]
	\end{theorem}
	% il gruppo di Galois della massima estensione abeliana di K è isomorfo al completamento profinito del gruppo moltiplcativo di K.
	\only<4->{
		Otteniamo una corrispondenza biunivoca
		\[
			\left\{ \substack{\text{estensioni abeliane} \\ \text{finite di } K} \right\} \longleftrightarrow \left\{ \substack{\text{sottogruppi} \\ \text{aperti di } \widehat{K^\times}} \right\}
		\]
	}

\end{frame}

% Che approccio prendiamo?
\begin{frame}{Coomologia di Gruppi}
	% Per poter studiare sistematicamente l'azione del gruppo di Galois assoluto sul gruppo additivo e quindi su quello moltiplicativo di \Ks, dato un gruppo G consideriamo
	% la categoria degli \Z[G] moduli, con oggetti i gruppi abeliani dotati di un'azione di G che rispetta l'operazione di A, e morfismi gli omomorfismi di gruppi che commutano con l'azione di G.
	\begin{itemize}
		\item<1-> Vogliamo studiare l'azione $\Gamma \curvearrowright \Kx$
		\item<2-> Dato $G$ gruppo, $\Gmod$ è la categoria degli $\Z[G]$-moduli
		      % Come in Teoria di Galois, consideriamo il funtore dei "punti fissi" dell'azione di G su A
		\item<3-> Il funtore
		      \[
			      \begin{aligned}
				      F:\Gmod & \longrightarrow \Ab                                                                          \\
				      A       & \longmapsto A^G = \{a\in A\mid g\cdot a=a\;\forall g\in G\} \only<4->{= \Hom_{\Z[G]}(\Z, A)}
			      \end{aligned}
		      \]
		      % Questo coincide con i morfismi di G-moduli da \Z con l'azione banale ad A, e quindi
		      \only<4->{è esatto a sinistra.}
	\end{itemize}
	% Quindi, definiamo la coomologia di G con coefficienti in A come i funtori derivati a destra di F:
	\begin{definition}<5->
		I gruppi di coomologia di $G$ con coefficienti in $A$ sono i funtori derivati
		\[
			\H{i}{G}{A}\coloneqq R^iF(A)
		\]
	\end{definition}
	% In particolare, data una successione esatta corta di G-moduli, otteniamo una successione esatta lunga di gruppi di coomologia.
	\begin{itemize}
		\item<5-> Per $G$ profinito, richiediamo inoltre $A=\bigcup_U A^U$, e possiamo ridurci al caso finito:
		      \[
			      \H{i}{G}{A}=\varinjlim_U\H{i}{G/U}{A^U}
		      \]
	\end{itemize}
\end{frame}

\begin{frame}{Coomologia di Galois}
	% Con questa tecnica, possiamo calcolare il primo gruppo di coomologia di \Kx:
	\begin{theorem}[Hilbert 90]<1->
		Sia $L/K$ estensione finita di campi. Allora, $\H{1}{\operatorname{Gal}(L/K)}{L^\times}=0$.
		Passando al limite:
		\[
			\H{1}{\Gamma}{\Kx}=0
		\]
	\end{theorem}
\end{frame}

\end{document}
