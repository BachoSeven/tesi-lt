\documentclass[a4paper]{report}

\usepackage[T1]{fontenc}
\usepackage[charter]{mathdesign}
% many useful symbols
\usepackage{textcomp}
\usepackage[english]{babel}
\usepackage{hyperref}
\usepackage{amsmath, amsthm}
% for \lightning
\usepackage{stmaryrd}
\usepackage{geometry}
\usepackage{tikz-cd}
\usepackage{bold-extra}
% for \coloneqq
\usepackage{mathtools}

% Bibliography
\usepackage[backend=biber, style=alphabetic]{biblatex}
\addbibresource{bibliography.bib}

% Remove indentation globally
\setlength{\parindent}{0pt}
% Have blank lines between paragraphs
\usepackage[parfill]{parskip}

\hypersetup{
    colorlinks = true, % links instead of boxes
    urlcolor   = cyan, % external hyperlinks
    linkcolor  = blue, % internal links
    citecolor  = cyan   % citations
}

% Definitions
\def\R{\mathbb{R}}
\def\C{\mathbb{C}}
\def\Q{\mathbb{Q}}
\def\N{\mathbb{N}}
\def\Z{\mathbb{Z}}
\def\K{K}
\def\Kx{\overline{\K}^\times}
\def\Qp{\mathbb{Q}_p}
\def\Zp{\mathbb{Z}_p}
\def\Gmod{\mathsf{Mod}_\mathsf{G}}
\def\Hmod{\mathsf{Mod}_\mathsf{H}}
\def\GHmod{\mathsf{Mod}_{\mathsf{G}/\mathsf{H}}}
\def\Gfmod{\mathsf{Mod}_{\mathsf{G}}^f}
\def\Ab{\mathsf{Ab}}

% Operators
\newcommand{\Gal}[1]{\mathrm{Gal}\left( #1^sep/#1 \right)}
\newcommand{\Aut}[1]{\mathrm{Aut}\left( #1 \right)}
\renewcommand{\H}[3]{H^{#1}( #2, \, #3 )}
\newcommand{\HH}[2]{H^2 #1, \, #2 )}
\newcommand{\Ind}[2]{\mathrm{I}_{#1}(#2)}
\DeclareMathOperator{\Hom}{Hom}
\DeclareMathOperator{\Res}{\mathtt{res}}
\DeclareMathOperator{\Cor}{\mathtt{cor}}
\DeclareMathOperator{\Inf}{\mathtt{inf}}
\DeclareMathOperator{\cd}{cd}
\DeclareMathOperator{\coker}{Coker}


\newcommand\Iso{\xrightarrow{
   \,\smash{\raisebox{-0.65ex}{\ensuremath{\scriptstyle\sim}}}\,}}

% use bullets for items
\renewcommand{\labelitemii}{$\circ$}
\renewcommand{\Im}{\operatorname{Im}}

\newcommand\numberthis{\addtocounter{equation}{1}\tag{\theequation}}

\theoremstyle{plain}
\newtheorem{theorem}{Theorem}[section]
\newtheorem{lemma}[theorem]{Lemma}

\theoremstyle{definition}
\newtheorem{definition}[theorem]{Definition}
\newtheorem{example}[theorem]{Example}

\theoremstyle{remark}
\newtheorem*{remark}{Remark}

\begin{document}

\begin{titlepage}

    \begin{figure}[!htb]
        \centering
        \includegraphics[scale=0.4]{cherubino.pdf}
    \end{figure}

    \begin{center}
        \textcolor[RGB]{2,87,144}{\textbf{\textsc{\huge Università di Pisa}}}\\
        \vspace{10mm}

        \large{\textsc{Dipartimento di Matematica}}\\
        \large{\textsc{Corso di Laurea Triennale in Matematica}}\\
        \vspace{35mm}
        {\Huge{\bf Local Tate Duality}}
    \end{center}
    \vspace{45mm}

    \begin{minipage}[t]{.5\textwidth}
        {\large{\scshape Candidato}{\normalsize\vspace{3mm}
                \bf\\ \large{Francesco Minnocci}}}
    \end{minipage}
    \hfill
    \begin{minipage}[t]{.5\textwidth}\raggedleft
        {\large{\scshape Relatore}{\normalsize\vspace{3mm} \bf\\ \large{Támas Szamuely}}}
    \end{minipage}

    \vspace{30mm}
    \centering{\large{\textsc{Anno Accademico 2023/2024}}}

\end{titlepage}

% make TOC
\tableofcontents
\newpage

% Notation, introduce as we go? things like Ab

\chapter*{Introduction}
\addcontentsline{toc}{section}{Introduction}

\chapter{Cohomology of Finite Groups}

\chapter{Cohomology of Profinite Groups}

\section{Cohomological Dimension}
\section{Galois Cohomology}
\subsection{Hilbert 90}
\subsection{Kummer Theory}
\subsection{Brauer Group of a Field}

\chapter{Local Fields}

\section{Structure of Local Fields}
\section{Computation of the Brauer Group}
\section{Cohomological Finiteness}

\chapter{Local Duality}

\section{Cup Product}

\section{Dualizing Module}

In this section we introduce the dualizing module, a central object for the study of local duality. While it can be defined explicitly (as done in e.g. \cite{Neukirch}),
we are going to prove its existence by homological methods and only compute it for the absolute Galois group of a p-adic field, which suffices to prove the duality theorem. We
start by introducing the type of duality we are interested in:

\begin{definition}
    If \(A\) is an abelian group, we defin \textit{Pontryagin dual} of \(A\) as

    \[
        A^*\coloneqq \Hom_{c}(A,\Q/\Z)
        .\]
    If \(A\) is torsion, \(A^*=\Hom(A,\Q/\Z)\) is profinite with the compact-open topology.
\end{definition}

\begin{remark}
    The functor \(A\mapsto A^*\)  induces an equivalence of categories

    \[
        \{ \text{Torsion abelian groups} \}^{\text{op}}  \cong \{ \text{Profinite abelian groups}, \}
    \]

    as a special case of Pontryagin duality for locally compact abelian groups.
    For a finite abelian group \(A\), the structure theorem yields a (non-canonical) isomorphism \(A\Iso A^*\), and by cardinality considerations we have the canonical isomorphism
    \begin{align*}
        A & \Iso A^{**}                           \\
        b & \mapsto \left( f\mapsto f(b) \right).
    \end{align*}

    Generalizing to torsion abelian groups, we can write such an \(A\) as the inductive limit of its finite subgroups \(A_i\) to get

    \[
        A^* = \Hom\left(\varinjlim_i A_i,\Q/\Z\right) = \varprojlim_i A_i^*
    \]

    which is a profinite group. Viceversa, if \(G\) is an abelian profinite group, we can write it as the projective limit of its finite quotients \(G_i\), and then

    \[
        G^* = \Hom_{c}\left( \varprojlim_i G_i,\Q/\Z \right) = \varinjlim_i G_i^*
    \]

    which is a torsion group. One then checks that this induces an equivalence of categories.
\end{remark}

\subsection{Existence}

Now, given a profinite group \(G\) of finite cohomological dimension \(\cd(G)=n\), denote by \(\Gfmod\) the category of finite discrete \(G\)-modules.
Then, by the previous remark we have a contravariant left-exact functor
\begin{align}\label{eq:CohomDual}
    \Gfmod & \to \Ab                \\
    A      & \mapsto \H{n}{G}{A}^*:
\end{align}
indeed, \(^*\) is an exact functor because it induces an equivalence, and using the long exact sequence we see that the \(n\)-th cohomology functor is right-exact, so their composition is left-exact by contravariance of \(^*\).

The dualizing module is defined through the following theorem, which gives sufficient conditions to the representability of this functor:

\begin{theorem}\label{thm:DualMod}
    Let \(G\) be a profinite group of finite cohomological dimension \(\cd(G)=n\) such that \(\H{n}{G}{A}\) is finite for all \(A\in\Gfmod\).

    Then, the functor \eqref{eq:CohomDual} is representable in the category \(\Gfmod\): there exists a torsion discrete \(G\)-module \(I\) together with a natural isomorphism
    \[
        \Hom_G\left( -,I \right) \simeq \H{n}{G}{-}^*
    \]
    of functors \(\Gfmod\to\Ab\). We say that \(I\) is the \textit{dualizing module} of \(G\).
\end{theorem}

The proof uses a lemma from homological algebra, for which we give a basic definition:

\begin{definition}
    A category \(\mathcal{C}\) is \textit{Noetherian} if:
    \begin{itemize}
        \item it is \textit{essentially small} (i.e. it's equivalent to a category whose objects form a set), and
        \item every object \(C\) of \(\mathcal{C}\) is \textit{Noetherian} (i.e. every ascending chain of subobjects of \(C\) stabilizes).
    \end{itemize}
\end{definition}

\begin{lemma}\label{lm:IndRep}
    Let \(\mathcal{C}\) be a Noetherian Abelian category, and \(F: \mathcal{C}\to\Ab\) a contravariant left-exact functor.
    Then, \(F\) is \textup{Ind}-representable: there exists a filtered inductive system \((I_j)\) of objects in \(\mathcal{C}\) such that \(F\) is naturally isomorphic to the functor
    \[A\mapsto \varinjlim_j \Hom(I_j,A).\]
\end{lemma}

We first show how this implies the existence of the dualizing module:

\begin{proof}[Proof of Theorem \eqref{thm:DualMod}]
    We apply Lemma \eqref{lm:IndRep} to the functor \(\H{n}{G}{-}^*\), which we have already shown to be left-exact; the category \(\Gfmod\) is Noetherian as it's small and its objects are finite.
    Thus, we obtain an inductive system \((I_j)\) and a natural isomorphism
    \[
        \H{n}{G}{-}^* \simeq \varinjlim_j \Hom(I_j,-).
    \]
    Now, define \(I\) to be the inductive limit of the \(I_j\). This is a discrete torsion \(G\)-module, and since \(A\) is finite, we have
    \[
        \Hom_G(A,I) \simeq \varinjlim_j \Hom(A,I_j) = \H{n}{G}{A}^*.
    \]
\end{proof}

Now we embark on the proof of Lemma \eqref{lm:IndRep}, which is
due to Grothendieck (in the case of an Artinian category, see section 3 of \cite{Grothendieck}):

\begin{proof}

\end{proof}

\subsection{Computation for p-adic Fields}

\section{Tate Duality}

Before proving Tate's local duality theorem for p-adic fields, we define
yet another type of duality for finite groups:

\begin{definition}
    Given a field \(\K\), let  $\Gamma$ be its absolute Galois group and $\mu$ the group of all the roots of unity in $\Kx$.
    For any finite \(\Gamma\) -module \(M\) whose torsion is prime to the characteristic of \(\K\),
    we define the \textbf{Cartier dual} of $M$ as
    \begin{equation*}
        M^{\prime} = \Hom_{\Z}(M, \Kx)=\Hom_{\Z}(M, \mu),
    \end{equation*}
    with the action of $\Gamma$ given by
    \begin{equation*}
        (\sigma \cdot f)(m) = \sigma(f(\sigma^{-1}\cdot m)).
    \end{equation*}
\end{definition}

\begin{theorem}
    Let $\K$ be a p-adic field, and $M$ a finite $\Gamma$-module.
    Then, for $i=0,1,2$ the cup product induces a perfect pairing of finite groups
    \begin{equation*}
        H^i(\K, M) \times H^{2-i}(\K, M^{\prime}) \to H^2(\K, \mu) \cong \Q/\Z.
    \end{equation*}
\end{theorem}
\begin{proof}
    We have already proved the finiteness of all the groups in \eqref{thm:FinCohom}.
    We start from $i=2$, where our computation of the dualizing module pays off: indeed, the content of Lemma \eqref{lm:CompatCupDual}
    in this situtation consists of the following commutative diagram
    \begin{equation*}
        \begin{tikzcd}
            {\H{0}{\K}{M^\prime}  \times \HH{\K}{M}} \ar[swap, rd, "\langle - {,} - \rangle_M "] \ar[r, "\cup"] & \HH{\K}{\mu} \ar[d, "i"] \\
            & \Q/\Z,
        \end{tikzcd}
    \end{equation*}
    where
    \[
        i = \langle \operatorname{Id}_{\Q/\Z}, - \rangle : \H{2}{\K}{\mu} \Iso \Q/\Z,
    \]
    which concludes as $\langle -{,}- \rangle_M$ is the perfect pairing given by the dualizing module.

    The case $i=0$ then follows by symmetry, as any finite $M$ can be identified with its double dual $M''$.
    We are now left with $i=1$: here it's enough to show that the map
    \[
        \H{1}{\K}{M} \overset{\cup}{\to} \H{1}{\K}{M^\prime}^*
    \]
    induced by the cup product is injective, as then applying the same argument to $M^\prime$ gets us surjectivity, since we are working with finite groups.
    To prove injectivity, we first find a finite $\Gamma$-module $B$ fitting in an exact sequence
    \begin{equation}\label{eq:InjSeq}
        0\to M\to B\to C\to 0
    \end{equation}
    such that the induced map
    \[
        \H{1}{\K}{B}\to\H{1}{\K}{B}
    \]
    is zero:
    as usual, we start by injecting $M$ in the induced module $\Ind{\Gamma}{M}$ which is torsion, and by the fundamental property \eqref{thm:IndLim} we have

    \[
        0=\H{n}{\K}{\Ind{\Gamma}{M}}=\varinjlim_{\substack{B\subset\Ind{\Gamma}{M} \\B\text{ finite}}}{\H{n}{\K}{B}},
    \]
    which implies the existence of a finite submodule \(B\) with \(\H{n}{\K}{B}=0\). As \(M\) is finite, we can find such a \(B\) which contains \(M\), and then
    the quotient \(C=B/M\) is finite as well.
    Now, we can relate the long exact sequences in cohomology associated with the short exact sequence \eqref{eq:InjSeq}
    and its dual using the cup product, which gets us the following commutative diagram (up to a sign, by \eqref{prop:CompatCupDiff}) with exact rows:
    \[
        \begin{tikzcd}
            \H{0}{\K}{B} \ar[r] \ar[d, "\cup"] & \H{0}{\K}{C} \ar[r] \ar[d, "\cup"] & \H{1}{\K}{M} \ar[d, "\cup"] \ar[r] & 0 \\
            \H{2}{\K}{B^\prime}^* \ar[r] & \H{2}{\K}{C^\prime}^* \ar[r] & \H{1}{\K}{M^\prime}^*.
        \end{tikzcd}
    \]
    As \(B\) and \(C\) are finite, the first two vertical maps are isomorphisms by the previous cases, and the third one is injective by diagram chasing.

\end{proof}

% Remark: The theorem also holds local fields of positive characteristic and p-torsion-free modules by <ref>

\chapter{Consequences}

\section{Local Class Field Theory}

% As a first application of local duality, we can prove the existence of the Artin map in the local case.

\nocite{*}
\printbibliography

\end{document}
