% remove the oneside option for printing
\documentclass[a4paper, oneside]{memoir}

\usepackage[T1]{fontenc}
% \usepackage{mlmodern}
% many useful symbols
\usepackage{textcomp}
\usepackage[english]{babel}
\usepackage{hyperref}
\usepackage{amsmath, amsthm, amssymb}
% for \lightning
\usepackage{stmaryrd}
\usepackage{color}
\usepackage{geometry}
\usepackage{tikz-cd}
\usepackage{bold-extra}
% for \coloneqq
\usepackage{mathtools}
\usepackage[capitalise]{cleveref}

% Bibliography
\usepackage[backend=biber, style=alphabetic]{biblatex}
\addbibresource{bibliography.bib}

% remove citations
\renewcommand{\cite}[1]{#1}
\renewcommand{\nocite}[1]{}
\def\printbibliography{}

\hypersetup{
    colorlinks = true, % links instead of boxes
    urlcolor   = cyan, % external hyperlinks
    linkcolor  = blue, % internal links
    citecolor  = cyan   % citations
}

% Highlight missing references in red
\makeatletter
\def\@setref#1#2#3{% csname, extract group, refname
    \ifx#1\relax
        \protect\G@refundefinedtrue
        \nfss@text{\reset@font\textcolor{red}{#3}}%
        \@latex@warning{%
            Reference `#3' on page \thepage \space undefined%
        }%
    \else
        \expandafter\Hy@setref@link#1\@empty\@empty\@empty\@nil{#2}%
    \fi
}
\makeatother

% % Have blank lines between paragraphs
% \nonzeroparskip
% % Remove indentation globally
% \setlength{\parindent}{0pt}

% Definitions
\def\R{\mathbb{R}}
\def\C{\mathbb{C}}
\def\Q{\mathbb{Q}}
\def\N{\mathbb{N}}
\def\Z{\mathbb{Z}}
\def\K{K}
\def\k{\kappa}
\def\Kx{\overline{\K}^\times}
\def\Ks{\overline{\K}}
\def\ks{\overline{\kappa}}
\def\Knr{\K_{\text{nr}}}
\def\Qp{\mathbb{Q}_p}
\def\Zp{\Z_p}
\def\Zn{\Z/n\Z}
\def\Gmod{\mathsf{Mod}_\mathsf{G}}
\def\Hmod{\mathsf{Mod}_\mathsf{H}}
\def\GHmod{\mathsf{Mod}_{\mathsf{G}/\mathsf{H}}}
\def\Gfmod{\mathsf{Mod}_{\mathsf{G}}^f}
\def\Ab{\mathsf{Ab}}

% Operators
\newcommand{\Gal}[1]{\mathrm{Gal}\left( #1_{s}/#1 \right)}
\newcommand{\Aut}[1]{\mathrm{Aut}\left( #1 \right)}
\renewcommand{\H}[3]{H^{#1}( #2, \, #3 )}
\newcommand{\HH}[2]{H^2(#1, \, #2 )}
\newcommand{\Ind}[2]{\mathrm{I}_{#1}(#2)}
\newcommand{\ZG}[1]{\Z[#1]}
\newcommand{\Br}[1]{\mathrm{Br}(#1)}
\DeclareMathOperator{\St}{\mathrm{St}}
\DeclareMathOperator{\Hom}{Hom}
\DeclareMathOperator{\Ext}{Ext}
\DeclareMathOperator{\Res}{\mathtt{res}}
\DeclareMathOperator{\Cor}{\mathtt{cor}}
\DeclareMathOperator{\Inf}{\mathtt{inf}}
\DeclareMathOperator{\cd}{cd}
\DeclareMathOperator{\coker}{Coker}
\DeclareMathOperator{\id}{id}
\DeclareMathOperator{\sh}{\mathtt{sh}}

\newcommand\Iso{\xrightarrow{
        \,\smash{\raisebox{-0.65ex}{\ensuremath{\scriptstyle\sim}}}\,}}

% Display math
\newcommand{\ssfrac}[2]{
    \raisebox{+0.3ex}{$#1$}
    /
    \raisebox{-0.3ex}{$#2$}
}
% Inline math
\newcommand{\sfrac}[2]{
    \raisebox{+0.3ex}{\scalebox{0.9}{$#1$}}
    /
    \raisebox{-0.3ex}{\scalebox{0.9}{$#2$}}
}

% use bullets for items
\renewcommand{\labelitemii}{$\circ$}
\renewcommand{\Im}{\operatorname{im}}

\newcommand\numberthis{\addtocounter{equation}{1}\tag{\theequation}}

\theoremstyle{plain}
\newtheorem{theorem}{Theorem}[section]
\newtheorem{lemma}[theorem]{Lemma}
\newtheorem{proposition}[theorem]{Proposition}
\newtheorem{corollary}[theorem]{Corollary}

\theoremstyle{definition}
\newtheorem{definition}[theorem]{Definition}
\newtheorem{example}[theorem]{Example}

\theoremstyle{remark}
\newtheorem{remark}[theorem]{Remark}

\chapterstyle{madsen}


\begin{document}

\begin{titlingpage}

    \begin{figure}[!htb]
        \centering
        \includegraphics[scale=0.4]{cherubino.pdf}
    \end{figure}

    \begin{center}
        \textcolor[RGB]{2,87,144}{\textbf{\textsc{\huge Università di Pisa}}}\\
        \vspace{10mm}

        \large{\textsc{Dipartimento di Matematica}}\\
        \large{\textsc{Corso di Laurea Triennale in Matematica}}\\
        \vspace{35mm}
        {\Huge{\bfseries Local Tate Duality}}
    \end{center}
    \vspace{45mm}

    \begin{minipage}[t]{.5\textwidth}
        {\large{\scshape Candidato}{\normalsize\vspace{3mm}
                \bfseries\\ \large{Francesco Minnocci}}}
    \end{minipage}
    \hfill
    \begin{minipage}[t]{.5\textwidth}\raggedleft
        {\large{\scshape Relatore}{\normalsize\vspace{3mm} \bfseries\\ \large{Támas Szamuely}}}
    \end{minipage}

    \vspace{30mm}
    \centering{\large{\textsc{Anno Accademico 2023/2024}}}

\end{titlingpage}

% make TOC
\tableofcontents

% TODO Notation, introduce as we go? things like Ab

\chapter*{Introduction}
\addcontentsline{toc}{section}{Introduction}

\chapter{Cohomology of Finite Groups}

\chapter{Cohomology of Profinite Groups}

\section{Cohomological Dimension}
\section{Galois Cohomology}
\subsection{Hilbert 90}
\subsection{Kummer Theory}
\subsection{Brauer Group of a Field}

\chapter{Local Fields}

\section{Structure of Local Fields}
\section{Computation of the Brauer Group}
\section{Cohomological Finiteness}

\chapter{Local Duality}

\section{Cup Product}
% TODO: define pairings, cup product, compatibility

\section{Dualizing Module}

In this section we introduce the dualizing module, a central object for the study of local duality. While it can be defined explicitly (as done in e.g. \cite{Neukirch}),
we are going to prove its existence by homological methods and only compute it for the absolute Galois group of a p-adic field, which suffices to prove the duality theorem. We
start by introducing the type of duality we are interested in:

\begin{definition}
    If \(A\) is an abelian group, we define the \textit{Pontryagin dual} of \(A\) as

    \[
        A^*\coloneqq \Hom_{c}(A,\Q/\Z)
        .\]
    If \(A\) is torsion, \(A^*=\Hom(A,\Q/\Z)\) is profinite with the compact-open topology.
\end{definition}

\begin{remark}
    The functor \(A\mapsto A^*\)  induces an equivalence of categories

    \[
        \{ \text{Torsion abelian groups} \}^{\text{op}}  \cong \{ \text{Profinite abelian groups}, \}
    \]

    as a special case of Pontryagin duality for locally compact abelian groups.
    For a finite abelian group \(A\), the structure theorem yields a (non-canonical) isomorphism \(A\Iso A^*\), and by cardinality considerations we have the canonical isomorphism
    \begin{align*}
        A & \Iso A^{**}                           \\
        b & \mapsto \left( f\mapsto f(b) \right).
    \end{align*}

    Generalizing to torsion abelian groups, we can write such an \(A\) as the inductive limit of its finite subgroups \(A_i\) to get

    \[
        A^* = \Hom(\varinjlim_i A_i,\Q/\Z) = \varprojlim_i A_i^*,
    \]

    which is a profinite group. Viceversa, if \(G\) is an abelian profinite group, we can write it as the projective limit of its finite quotients \(G_i\), and then

    \[
        G^* = \Hom_{c}( \varprojlim_i G_i,\Q/\Z ) = \varinjlim_i G_i^*
    \]

    is a torsion group. One checks that this actually induces an equivalence of categories, for which we refer to \cite{RibesZalesskii}, \S 2.9.
\end{remark}

\subsection{Existence}

Now, given a profinite group \(G\) of finite cohomological dimension \(\cd(G)=n\), denote by \(\Gfmod\) the category of finite discrete \(G\)-modules.
Then, by the previous remark we have a contravariant left-exact functor
\begin{align}\label{eq:CohomDual}
    \Gfmod & \to \Ab                \\
    A      & \mapsto \H{n}{G}{A}^*:
\end{align}
indeed, \(^*\) is an exact functor because it induces an equivalence, and using the long exact sequence we see that the \(n\)-th cohomology functor is right-exact, so their composition is left-exact by contravariance of \(^*\).

The dualizing module is defined through the following theorem, which gives sufficient conditions to the representability of this functor:

\begin{theorem}\label{thm:DualMod}
    Let \(G\) be a profinite group of finite cohomological dimension \(\cd(G)=n\) such that \(\H{n}{G}{A}\) is finite for all \(A\in\Gfmod\).

    Then, the functor \eqref{eq:CohomDual} is representable in the category \(\Gfmod\): there exists a torsion discrete \(G\)-module \(I\) together with a natural isomorphism
    \begin{equation}\label{eq:FuncIso}
        \Hom_G\left( -,I \right) \simeq \H{n}{G}{-}^*
    \end{equation}
    of functors \(\Gfmod\to\Ab\).
\end{theorem}

\begin{definition}
    In the situation of Theorem \ref{thm:DualMod}, \(I\) is called the \textit{dualizing module} of \(G\).
\end{definition}

The proof uses a lemma from homological algebra, for which we give a basic definition:

\begin{definition}
    A category \(\mathcal{C}\) is \textit{Noetherian} if:
    \begin{itemize}
        \item it is \textit{essentially small} (i.e. it's equivalent to a category whose objects form a set), and
        \item every object \(C\) of \(\mathcal{C}\) is \textit{Noetherian} (i.e. every ascending chain of subobjects of \(C\) stabilizes).
    \end{itemize}
\end{definition}

\begin{lemma}\label{lm:IndRep}
    Let \(\mathcal{C}\) be a Noetherian Abelian category, and \(F: \mathcal{C}\to\Ab\) a contravariant left-exact functor.
    Then, \(F\) is \textup{Ind}-representable: there exists a filtered inductive system \((I_j)\) of objects in \(\mathcal{C}\) such that \(F\) is naturally isomorphic to the functor
    \[A\mapsto \varinjlim_j \Hom(A, I_j).\]
\end{lemma}

We first show how this implies the existence of the dualizing module:
\begin{proof}[Proof of Theorem \ref{thm:DualMod}]
    We apply Lemma \ref{lm:IndRep} to the functor \(\H{n}{G}{-}^*\), which we have already shown to be left-exact; the category \(\Gfmod\) is Noetherian as it's small and its objects are finite.
    Thus, we obtain an inductive system \((I_j)\) and a natural isomorphism
    \[
        \H{n}{G}{-}^* \simeq \varinjlim_j \Hom(-,I_j).
    \]
    Now, set \(I\coloneqq \varinjlim_j I_j\). This is a discrete torsion \(G\)-module, and since \(A\) is finite we conclude:
    \begin{align*}
        \Hom_G(A,I) & \simeq \varinjlim_j \Hom(A,I_j) \\
                    & = \H{n}{G}{A}^*.
    \end{align*}
\end{proof}

\begin{remark}\label{rm:Torsion}
    We can generalize Theorem \ref{thm:DualMod} to discrete torsion \(G\)-modules, just by writing such an \(A\) as the inductive limit of its finite submodules \(A=\displaystyle\varinjlim_{\mathclap{\substack{B\subset A \\B\text{ finite}}}}{B}\):
    \begin{align*}
        \H{n}{G}{A}^* & =\varprojlim \H{n}{G}{B}^*     \\
                      & \simeq \varprojlim \Hom_G(B,I) \\
                      & = \Hom_G(A,I).
    \end{align*}
\end{remark}
\begin{remark}\label{rm:pTorsion}
    If we only consider profinite groups of finite \(p\)-cohomological dimension \(n\), the analogue of Theorem \ref{thm:DualMod} holds with the same proof, provided we further restrict ourselves to \(p\)-primary torsion modules (as the \(n\)-th cohomology functor is right-exact on the corresponding subcategory)
\end{remark}

Let us now embark on the proof of Lemma \ref{lm:IndRep}, which is
due to Grothendieck (in the case of an Artinian category, see \cite{Grothendieck} \S 3):
\begin{proof}
    A pair \((A,x)\), for \(A\) in \(\mathcal{C}\) and \(x\) in \(F(A)\), is called
    \textbf{minimal} if \(x\notin F(B)\) for each surjection \(B\twoheadrightarrow A\) with a non-trivial kernel. This makes sense as for \(F\) left-exact we can view \(F(B)\) as a subobject of \(F(A)\); we shall use this repeatedly in the following.

    Given two pairs \((A,x)\) and \((B,y)\), we say that \((A,x)\) \textbf{dominates} \((B,y)\) if there is a morphism \(p:A\to B\) such that \(F(p)(y)=x\).

    Using the Noetherian hypothesis, we can show that every pair \((A,x)\) is dominated by a minimal pair: to construct it, consider the poset \(\Sigma\) of all subjobects \(A_0 \subset A\) such that \((A/A_0,y)\) dominates \((A,x)\) for some \(y\in F(A/A_0)\)  through the projection morphism \(p:\,A\twoheadrightarrow A/A_0\).
    Taking \(A_0=0\) and \(y=x\) shows that \(\Sigma\) is non-empty: indeed, \(p=\operatorname{id}_A\) and \(F(p)(y)=\operatorname{id}_{F(A)}(y)=x\). Then, by the Noetherian condition \(\Sigma\) admits a maximal element \(A_0\), and we show that \((A/A_0,y)\) is a minimal pair: given \(p^{\prime} :B\twoheadrightarrow A_0\) with \(\ker(p^{\prime})\neq 0\), if \(y\) is in \(F(B)\) we can consider the pair \((B,y)\) and the morphism \(q\coloneqq p^{\prime} \circ p: A\twoheadrightarrow B\). Then, \((B,y)\) dominates \((A,x)\) since
    \[F(q)(y)=F(p)(F(p^{\prime} )(y))=F(p)(y),\]  but then \(B= (A/A_0)/\ker(p^{\prime})\) is a quotient of \(A\) which contradicts the maximality of \(A_0\).

    Furthermore, if \((A,x)\) is dominated by a minimal pair \((B,y)\), we claim that there is a \textit{unique} morphism \(p:A\to B\) such that \(F(p)(y)=x\).
    Indeed, let \(q:A\to B\) be a morphism with \(F(q)(y)=x\), then \[F(p-q)(y)= 0.\]
    Moreover, the surjection \(A\twoheadrightarrow\Im(p-q)\) induces an injective morphism \(F(\Im(p-q))\hookrightarrow F(A)\), and composing it with the inclusion \(i: \Im(p-q)\hookrightarrow A\) we get a morphism \[F(B)\overset{F(i)}{\to} F(\Im(p-q))\hookrightarrow F(A)\] which sends \(y\) to \(0\), and conclude that \(x\in\ker(F(i))\).

    Finally, taking \(C\) to be the cokernel of \((p-q)\) yields an exact sequence
    \begin{equation}\label{eq:MinSeq}
        0\to\Im(p-q)\overset{i}{\to} B\to C\to 0
    \end{equation}
    which in turn induces an exact sequence
    \[0\to F(C)\to F(B) \to F(\Im(p-q)),\]
    which shows that \(\ker(F(i))=F(C)\), so \(x\) is contained in \(F(C)\). However, by minimality of \((B,y)\) the second
    map in \eqref{eq:MinSeq} is an isomorphism, which means that \(p=q\).

    The set of minimal pairs can be ordered by setting \((A,x)\leq (B,y)\) if \((A,x)\) dominates \((B,y)\). This defines an inductive system \((I_j,x_j)\), as any two minimal pairs \((I_j,x_j)\) and \((I_k,x_k)\) are dominated by their direct sum \(\left(I_j\oplus I_k,(x_j,x_k)\right)\), which is itself dominated by a minimal pair.

    We thus get a canonical element \(x\coloneqq (x_j)\) of \(F(I)\coloneqq \varprojlim_j F(I_j)\), which we use to define a functorial homomorphism

    \[
        \phi: \; \varinjlim_j \Hom(A,I_j)\longrightarrow F(A),
    \]

    by sending \(f\coloneqq (f_j)\) to \(F(f)(x)\). This is well defined because \(\Hom\) preserves limits, and so
    \[
        F(f)\in\varinjlim_j{\Hom(F(I_j),F(A))}=\Hom(F(I),F(A)).
    \]

    Finally, we show that \(\phi\) is an isomorphism: if \((f_j)\) is sent to \(0\), then for any \(j\) the two morphisms
    \[
        F(f_j), F(0):\; F(I_j) \to F(A)
    \]
    both send \(x_j\) to \(0.\) Since \((I_j,x_j)\) is a minimal a pair which dominates \((A,0)\), we deduce that \(f_j\) must be the zero morphism, and by arbitrariety of \(j\) we get injectivity.

    For surjectivity, given any \(y\in F(A)\) we know that \((A,y)\) is dominated by a minimal pair \((I_j,x_j),\) so there is a unique \(f_j:\, A\to I_j\) satisfying \(F(f_j)(x_j)=y.\) Then, the limit of the \(f_j\) is sent by \(\phi\) to \(y,\) and we are done.
\end{proof}


\begin{remark}
    We can view the functorial isomorphism \eqref{eq:FuncIso} as a \textbf{perfect pairing} of \textit{torsion} groups (in light of Remark \ref{rm:Torsion}, which in particular applies to $A=I$):
    \[
        \langle-,-\rangle_A:\,\Hom_G(A,I)\times\H{n}{G}{A}\longrightarrow\Q /\Z.
    \]
\end{remark}

\begin{definition}
    For any \(G\)-module \(A\) , we define the \textit{Cartier dual} of \(A\) as
    \[
        A^{\prime}\coloneqq \Hom_\Z(A,I),
    \]
    which we make into a \(G\)-module by setting
    \[
        (\sigma\cdot f)(a)=\sigma(f(\sigma^{-1}\cdot a)).
    \]

    This yields a pairing
    \begin{align*}
        A^\prime\times A & \to I         \\
        (f,a)            & \mapsto f(a),
    \end{align*}
    and thus a cup product
    \[
        \H{0}{G}{A^\prime}\times\H{n}{G}{A}\overset{\cup}{\longrightarrow}\H{n}{G}{I}.
    \]

    This is related to the dual of cohomology by
    \[
        \H{0}{G}{A^{\prime}} = \Hom_G(A,I):
    \]
    indeed, \(\H{0}{G}{A^{\prime} }\) are the \(G\)-invariants of \(A^{\prime} = \Hom_\Z(A,I)\), so that
    \[
        \H{0}{G}{A^{\prime}} = (\Hom_\Z(A,I))^G = \Hom_G(A,I).
    \]
\end{definition}

We will need the following compatibility of the cup product with the pairing pairing:

\begin{lemma}\label{lm:CompatCupDual}
    In the situation of the previous definition, the pairing \(\langle-,- \rangle_A\) factors through the cup product and the homomorphism \(i\coloneqq \langle \operatorname{id}_I,-\rangle_I: \H{n}{G}{I}\to \Q/\Z\), i.e. the following diagram commutes:
    \[
        \begin{tikzcd}
            {\H{0}{G}{A^\prime}  \times \H{n}{G}{A}} \ar[swap, rd, "\langle - {,} - \rangle_A "] \ar[r, "\cup"] & \H{n}{G}{I} \ar[d, "i"] \\
            & \Q/\Z.
        \end{tikzcd}
    \]
\end{lemma}

\begin{proof}
    Any morphism \(f\in\Hom_G(A,I)\) induces a homomorphism in cohomology
    \[
        f^*:\, \H{n}{G}{A}\to\H{n}{G}{I},
    \]
    and by composition on the left a homomorphism
    \[
        f_*:\, \Hom_G(I,I)\to\Hom_G(A,I)
        .\]
    Now, writing out the functorality of the dualizing module gives the following commutative diagram:
    \[
        \tikzset{
            symbol/.style={
                    draw=none,
                    every to/.append style={
                            edge node={node [sloped, allow upside down, auto=false]{$#1$}}}
                }
        }
        \begin{tikzcd}
            \Hom_G(A,I) \arrow[r,symbol=\times] &[-2em] \ar[dd, "f^*"] \H{n}{G}{A} \ar[rd, "\langle - {,} - \rangle_A "] &[2em] \\
            && \Q/\Z \\
            \Hom_G(I,I)  \arrow[r,symbol=\times] \ar[uu, "f_*"] & \H{n}{G}{I}  \ar[swap, ru, "\langle - {,} - \rangle_I "]
        \end{tikzcd}
    \]

    As remarked in \ref{rm:cup0n}, here the cup product of \(f\in\H{0}{G}{A^{\prime}}\) with some \(\alpha\in\H{n}{G}{A}\) is given by
    \( f^*(\alpha) \). Since \(f_*\) clearly sends \(\operatorname{id}_I\) to \(f\), the above diagram implies that
    \[
        \langle f, \alpha \rangle_A = \langle \operatorname{id}_I, f^*(\alpha)\rangle_I=i(f\cup\alpha).
    \]
\end{proof}

\begin{corollary}
    It follows that the pair \((I,i)\) is unique up to unique isomorphism.
\end{corollary}

Before moving on to the computation of the dualizing module of the absolute Galois group of a \(p\)-adic field, we need a lemma about dualizing modules of open subgroups:

\begin{lemma}\label{lm:dualOpenSgr}
    Let \(G\) be a profinite group of finite cohomological dimension \(n\) such that \(\H{n}{G}{A}\) is finite for all \(A\in\Gfmod\). If \(U\subset G\) is an open subgroup and \(I\) is the dualizing module of \(G\), then \(I\) (viewed as an \(U\)-module) is the dualizing module of \(U\), and the homomorphism
    \[
        \H{n}{U}{A}\to\H{n}{G}{A}
    \]
    defined by dualizing the inclusion \(i:\Hom_G(A,I) \hookrightarrow \Hom_U(A,I) \) is simply the corestriction.
\end{lemma}

\begin{proof}
    As \(U\) is open and \(\cd(G)<\infty\), by Proposition \ref{ineqCd} we get \(\cd(U)=\cd(G)\). By uniqueness of the dualizing module, it's enough to show that \(\H{n}{U}{-}^*\simeq\Hom_U(-,I)\), and then the first claim follows from the functorial isomorphisms
    \begin{align*}
        \H{n}{U}{A}^* & \simeq \H{n}{G}{\mathrm{I}_G^U(A)}   & \text{(Shapiro)}                               \\
                      & \simeq \Hom_G(\mathrm{I}_G^U(A),\,I) & \text{($I$ is the dualizing module of $G$)}    \\
                      & \simeq \Hom_U(A,\,I)                 & \text{(by Proposition \ref{prop:IndLeftAdj})}.
    \end{align*}
    For the second claim, recall from remark \ref{rm:Cor} that the corestriction homomorphism is induced by the surjective morphism of \(G\)-modules
    \[\mathrm{I}_G^U(A) \overset{\pi}{\longrightarrow} A\]
    defined as
    \[ \pi(f)= \sum_{g\in G/U}{g\cdot f(g^{-1})}.\]
    On the other hand, $\pi$ induces a morphism of modules \(\Hom_G(A,I)\overset{\pi_*}{\longrightarrow}\Hom_G(\mathrm{I}_G^U(A),I)\).
    Using the functoriality of the dualizing module and the Shapiro isomorphism $\operatorname{Sh}$ (\eqref{eq:Shapiro}),
    we get the following commutative diagram:
    \[
        \begin{tikzcd}[column sep=small, row sep=large]
            \Hom_U(A,I)
            \rar["\Phi"] &
            \Hom_G(\mathrm{I}_G^U(A), I)
            \rar &
            \H{n}{G}{\mathrm{I}_G^U(A)}^*
            \rar["(-)^*"]&\H{n}{G}{\mathrm{I}_G^U(A)}
            \rar["\operatorname{Sh}\ "]
            \ar[d, "\Cor"]&\H{n}{U}{A}\ar[dl] \\
            &
            \Hom_G(A,I) \ar[swap, lu, "i"]
            \ar[u, "\pi_*"]
            \rar &
            \H{n}{G}{A}^*
            \ar[u,"\pi_n^*"]
            \rar["(-)^*"] &
            \H{n}{G}{A} \\
        \end{tikzcd}
    \]
    Here, $\Phi$ is the isomorphism of Proposition \ref{prop:IndLeftAdj},
    which makes the leftmost triangle commute as
    \[
        \pi_*(\varphi)(f) = \varphi\left(\sum\nolimits_{g\in G/U}\;{g\cdot f(g^{-1} )}\right) = \sum\nolimits_{g\in G/U}\;{g\cdot \varphi(f(g^{-1}))} = (\Phi\circ i) (\varphi)(f)
    \]
    for any \(\varphi\in\Hom_G(A,I)\) and \(f\in\mathrm{I}_G^U(A)\).
\end{proof}

\subsection{Computation for p-adic Fields}

Finally, we apply the theory of dualizing modules to the absolute Galois group \(\Gamma\) of a \(p\)-adic field \(\K\): by \ref{thm:CohomFin} we know that \(\cd(\Gamma)=2\), and by \ref{thm:FinCohom} the groups \(\H{2}{\K}{A}\) are finite for any finite \(\Gamma_{\K}\)-module $A$. Thus, by Theorem \ref{thm:DualMod} \(\Gamma\) has a dualizing module \(I\), which we now compute.

\begin{proposition}
    The dualizing module of \(\Gamma\) is canonically isomorphic to the \(\Gamma\)-module of roots of unity \(\mu\subset\Kx\).
\end{proposition}

\begin{proof}
    Let $I_n\coloneqq I[n]$ be the kernel of multiplication by $n$ on $I$, and take an open subgroup $U\subset\Gamma$. By Lemma \ref{lm:dualOpenSgr}, the dualizing module of $U$ is the same as that of $\Gamma$.

    Moreover, by \ref{thm:H2mu} we know that \(\H{2}{U}{\mu_n}^*\simeq\Zn\), and by \ref{thm:padicBrauer} the corestriction $\Br{K^U}\to\Br{K}$ is the identity of $\Q/\Z$.

    Thus, if $U\subset V$ are open subgroups of $\Gamma$, the functoriality of the dualizing module yields the following commutative diagram:
    \[
        \begin{tikzcd}
            \Hom_V(\mu_n,I_n)
            \ar[d]
            \ar[r] &
            \H{2}{V}{\mu_n}^*
            \ar[d, "\Cor^*=\,\operatorname{id}"]
            \rar["\sim"] &
            \Zn
            \ar[equal, d] \\
            \Hom_U(\mu_n,I_n)
            \ar[r] &
            \H{2}{U}{\mu_n}^*
            \rar["\sim"] &
            \Zn,
        \end{tikzcd}
    \]
    where the middle map is the is the corestriction by the second part of Lemma \ref{lm:dualOpenSgr}.
    This shows that \(\H{2}{U}{\mu_n}\) doesn't depend on \(U\), and by Remark \ref{rm:discGmodUnion} we get
    \[
        \Hom_\Z(\mu_n,I_n)=\bigcup_{\mathclap{\substack{U\subset\,\Gamma\\U\text{ open}}}}{\Hom_U(\mu_n,I_n)}\simeq\Zn
    \]
    In particular, taking $U$ to be $\Gamma$ tells us that $\Gamma$ acts trivially on $\Hom_\Z(\mu_n,I_n)$, or equivalently that any homomorphism $\mu_n\to I_n$ is $\Gamma$-equivariant.

    Let $f_n$ be the canonical generator of $\Hom_\Z(\mu_n,I_n)$ associated with $1\in\Zn$. Then $f_n$ is injective because it has order $n$, and surjective because otherwise any element outside its image wouldn't be reached by any element of $\langle f_n\rangle=\Hom_\Z(\mu_n,I_n)$, but as $I_n$ has exponent $n$ there are homomorphisms $\mu_n\to I_n$ sending a generator of $\mu_n$ to any element of $I_n$.
\end{proof}

\section{Tate Duality}

Before proving Tate's local duality theorem for p-adic fields, we define
yet another type of duality for finite groups:

\begin{definition}
    Given a field \(\K\), let  $\Gamma$ be its absolute Galois group and $\mu$ the group of all the roots of unity in $\Kx$.
    For any finite \(\Gamma\) -module \(M\) whose torsion is prime to the characteristic of \(\K\),
    we define the \textbf{Cartier dual} of $M$ as
    \begin{equation*}
        M^{\prime} = \Hom_{\Z}(M, \Kx)=\Hom_{\Z}(M, \mu),
    \end{equation*}
    with the action of $\Gamma$ given by
    \begin{equation*}
        (\sigma \cdot f)(m) = \sigma(f(\sigma^{-1}\cdot m)).
    \end{equation*}
\end{definition}

\begin{theorem}
    Let $\K$ be a p-adic field, and $M$ a finite $\Gamma$-module.
    Then, for $i=0,1,2$ the cup product induces a perfect pairing of finite groups
    \begin{equation*}
        H^i(\K, M) \times H^{2-i}(\K, M^{\prime}) \to H^2(\K, \mu) \cong \Q/\Z.
    \end{equation*}
\end{theorem}
\begin{proof}
    We have already proved the finiteness of all the groups in \ref{thm:FinCohom}.
    We start from $i=2$, where our computation of the dualizing module pays off: indeed, the content of Lemma \ref{lm:CompatCupDual}
    in this situtation consists of the following commutative diagram
    \begin{equation*}
        \begin{tikzcd}
            {\H{0}{\K}{M^\prime}  \times \HH{\K}{M}} \ar[swap, rd, "\langle - {,} - \rangle_M "] \ar[r, "\cup"] & \HH{\K}{\mu} \ar[d, "i"] \\
            & \Q/\Z,
        \end{tikzcd}
    \end{equation*}
    where
    \[
        i = \langle \operatorname{id}_{\Q/\Z}, - \rangle : \H{2}{\K}{\mu} \Iso \Q/\Z.
    \]
    Since \(\langle -{,}- \rangle_M\) is a perfect pairing, we are done.

    The case $i=0$ then follows by symmetry, as any finite $M$ can be identified with its double dual $M''$.
    We are now left with $i=1$: here it's enough to show that the map
    \[
        \H{1}{\K}{M} \overset{\cup}{\to} \H{1}{\K}{M^\prime}^*
    \]
    induced by the cup product is injective, as then applying the same argument to $M^\prime$ gets us surjectivity, since we are working with finite groups.
    To prove injectivity, we first find a finite $\Gamma$-module $B$ fitting in an exact sequence
    \begin{equation}\label{eq:InjSeq}
        0\to M\to B\to C\to 0
    \end{equation}
    such that the induced map
    \[
        \H{1}{\K}{B}\to\H{1}{\K}{B}
    \]
    is zero:
    as usual, we start by injecting $M$ in the induced module $\Ind{\Gamma}{M}$ which is torsion, and by the fundamental property \ref{thm:IndLim} we have

    \[
        0=\H{n}{\K}{\Ind{\Gamma}{M}}=\varinjlim_{\substack{B\subset\Ind{\Gamma}{M} \\B\text{ finite}}}{\H{n}{\K}{B}},
    \]
    which implies the existence of a finite submodule \(B\) with \(\H{n}{\K}{B}=0\). As \(M\) is finite, we can find such a \(B\) which contains \(M\), and then
    the quotient \(C=B/M\) is finite as well.
    Now, we can relate the long exact sequences in cohomology associated with the short exact sequence \eqref{eq:InjSeq}
    and its dual using the cup product, which gets us the following commutative diagram (up to a sign, by \ref{prop:CompatCupDiff}) with exact rows:
    \[
        \begin{tikzcd}
            \H{0}{\K}{B} \ar[r] \ar[d, "\cup"] & \H{0}{\K}{C} \ar[r] \ar[d, "\cup"] & \H{1}{\K}{M} \ar[d, "\cup"] \ar[r] & 0 \\
            \H{2}{\K}{B^\prime}^* \ar[r] & \H{2}{\K}{C^\prime}^* \ar[r] & \H{1}{\K}{M^\prime}^*.
        \end{tikzcd}
    \]
    As \(B\) and \(C\) are finite, the first two vertical maps are isomorphisms by the previous cases, and the third one is injective by diagram chasing.

\end{proof}

% TODO Remark: The theorem also holds local fields of positive characteristic and p-torsion-free modules by <ref>

\chapter{Consequences}

\section{Local Class Field Theory}

% For the chain of isomorphisms, need: remark that the bidual of a (non-necessarily abelian) profinite group is isomorphic to the original group.

\nocite{*}
\printbibliography

\end{document}