\documentclass[a4paper]{article}

\usepackage[T1]{fontenc}
% many useful symbols
\usepackage{textcomp}
\usepackage[english]{babel}
\usepackage{hyperref}
\usepackage{amsmath, amssymb, amsthm}
% for \lightning
\usepackage{stmaryrd}
\usepackage{geometry}
\usepackage{tikz-cd}
\usepackage{bold-extra}
% for \coloneqq
\usepackage{mathtools}

% Bibliography
\usepackage[backend=biber, style=alphabetic]{biblatex}
\addbibresource{bibliography.bib}

% Remove indentation globally
\setlength{\parindent}{0pt}
% Have blank lines between paragraphs
\usepackage[parfill]{parskip}

\hypersetup{
    colorlinks = true, % links instead of boxes
    urlcolor   = cyan, % external hyperlinks
    linkcolor  = blue, % internal links
    citecolor  = cyan   % citations
}

% Definitions
\def\R{\mathbb{R}}
\def\C{\mathbb{C}}
\def\Q{\mathbb{Q}}
\def\N{\mathbb{N}}
\def\Z{\mathbb{Z}}
\def\K{K}
\def\Kx{\overline{\K}^\times}
\def\Qp{\mathbb{Q}_p}
\def\Zp{\mathbb{Z}_p}
\def\Gmod{\mathsf{Mod}_\mathsf{G}}
\def\Hmod{\mathsf{Mod}_\mathsf{H}}
\def\GHmod{\mathsf{Mod}_{\mathsf{G}/\mathsf{H}}}

% Operators
\newcommand{\Gal}[1]{\mathrm{Gal}\left( #1^sep/#1 \right)}
\newcommand{\Aut}[1]{\mathrm{Aut}\left( #1 \right)}
\renewcommand{\H}[3]{H^{#1}\left( #2, \, #3 \right)}
\newcommand{\HH}[2]{H^2\left( #1, \, #2 \right)}
\newcommand{\Ind}[2]{\mathrm{I}_{#1}(#2)}
\DeclareMathOperator{\Hom}{Hom}
\DeclareMathOperator{\Res}{\mathtt{res}}
\DeclareMathOperator{\Cor}{\mathtt{cor}}
\DeclareMathOperator{\Inf}{\mathtt{inf}}
\DeclareMathOperator{\cd}{cd}
\DeclareMathOperator{\coker}{Coker}


\newcommand\Iso{\xrightarrow{
   \,\smash{\raisebox{-0.65ex}{\ensuremath{\scriptstyle\sim}}}\,}}

% use bullets for items
\renewcommand{\labelitemii}{$\circ$}
\renewcommand{\Im}{\operatorname{Im}}

\newcommand\numberthis{\addtocounter{equation}{1}\tag{\theequation}}

\newtheorem{theorem}{Theorem}[section]
\newtheorem{lemma}[theorem]{Lemma}

\theoremstyle{definition}
\newtheorem{definition}[theorem]{Definition}

\theoremstyle{definition}
\newtheorem{example}[theorem]{Example}

\theoremstyle{remark}
\newtheorem*{remark}{Remark}

\begin{document}

\begin{titlepage}

    \begin{figure}[!htb]
        \centering
        \includegraphics[scale=0.4]{cherubino.pdf}
    \end{figure}

    \begin{center}
        \textcolor[RGB]{2,87,144}{\textbf{\textsc{\huge Università di Pisa}}}\\
        \vspace{10mm}

        \large{\textsc{Dipartimento di Matematica}}\\
        \large{\textsc{Corso di Laurea Triennale in Matematica}}\\
        \vspace{35mm}
        {\Huge{\bf Local Tate Duality}}
    \end{center}
    \vspace{45mm}

    \begin{minipage}[t]{.5\textwidth}
        {\large{\scshape Candidato}{\normalsize\vspace{3mm}
                \bf\\ \large{Francesco Minnocci}}}
    \end{minipage}
    \hfill
    \begin{minipage}[t]{.5\textwidth}\raggedleft
        {\large{\scshape Relatore}{\normalsize\vspace{3mm} \bf\\ \large{Támas Szamuely}}}
    \end{minipage}

    \vspace{30mm}
    \centering{\large{\textsc{Anno Accademico 2023/2024}}}

\end{titlepage}

% make TOC
\tableofcontents
\newpage

\section*{Introduction}
\addcontentsline{toc}{section}{Introduction}

\section{Cohomology of Finite Groups}

\section{Cohomology of Profinite Groups}

\subsection{Cohomological Dimension}
\subsection{Galois Cohomology}
\subsubsection{Hilbert 90}
\subsubsection{Kummer Theory}
\subsubsection{Brauer Group of a Field}

\section{Local Fields}

\subsection{Structure of Local Fields}
\subsection{Computation of the Brauer Group}
\subsection{Cohomological Finiteness}

\section{Local Duality}

\subsection{Cup Product}

\subsection{Dualizing Module}

In this section we introduce the dualizing module, a central object for the study of local duality. While it can be defined explicitly (as done in e.g. \cite{Neukirch}),
whereas we are going to prove its existence by homological methods and only compute it for the absolute group of a p-adic field, which is faster and suffices to prove the duality theorem.

Recall that, if \(B\) is a torsion abelian group, the \textit{Pontryagin dual} of \(B\) is

\[
    B^*\coloneqq \Hom(B,\Q/\Z).
\]

The functor \(B\mapsto B^*\)  induces an equivalence of categories

\[
    \{ \text{Torsion abelian groups} \}^{\text{op}}  \cong \{ \text{Profinite abelian groups}, \}
\]

as a special case of Pontryagin duality for locally compact abelian groups.

Indeed, for a finite abelian group \(B\), the structure theorem yields a (non-canonical) isomorphism \(B\Iso B^*\), so that by cardinality considerations we have the canonical isomorphism
\begin{align*}
    B & \Iso B^{**}                           \\
    b & \mapsto \left( f\mapsto f(b) \right).
\end{align*}

For the generalization to torsion abelian groups, writing such a \(B\) as the inductive limit of its finite subgroups \(B_i\) we obtain

\[
    B^* = \varprojlim_i B_i^*,
\]

which is profinite, and viceversa if \(G\) is profinite, we can write it as the projective limit of its finite quotients \(G_i\), and then

\[
    G^* = \varinjlim_i G_i^*,
\]

which is a torsion group.

and for profinite abelian groups, by the structure theorem we have \(B\Iso B^{**}\).

Given a profinite group \(G\)  of finite cohomological dimension \(\cd(G)=n<\infty,\)

\subsubsection{Existence}
\subsubsection{Computation for p-adic Fields}

\subsection{Tate Duality}

We can now prove the main result: Tate's local duality theorem for p-adic fields.

\begin{definition}
    Given a field \(\K\), let  $\Gamma$ be its absolute Galois group and $\mu$ the group of all the roots of unity in $\Kx$.
    For any finite \(\Gamma\) -module \(M\) whose torsion is prime to the characteristic of \(\K\),
    we define the \textbf{Cartier dual} of $M$ as
    \begin{equation*}
        M^{\prime} = \Hom_{\Z}(M, \Kx)=\Hom_{\Z}(M, \mu),
    \end{equation*}
    with the action of $\Gamma$ given by
    \begin{equation*}
        (\sigma \cdot f)(m) = \sigma(f(\sigma^{-1}\cdot m)).
    \end{equation*}
\end{definition}

\begin{theorem}
    Let $\K$ be a p-adic field, and $M$ a finite $\Gamma$-module.
    Then, for $i=0,1,2$ the cup product induces a perfect pairing of finite groups
    \begin{equation*}
        H^i(\K, M) \times H^{2-i}(\K, M^{\prime}) \to H^2(\K, \mu) \cong \Q/\Z.
    \end{equation*}
\end{theorem}
\begin{proof}
    We have already proved the finiteness of all the groups in \eqref{finCohom}.
    We start from $i=2$, where our computation of the dualizing module pays off: indeed, the content of Lemma \eqref{compDualCup}
    in this situtation consists of the following commutative diagram
    \begin{equation*}
        \begin{tikzcd}
            {\H{0}{\K}{M^\prime}  \times \HH{\K}{M}} \ar[swap, rd, "\langle - {,} - \rangle_M "] \ar[r, "\cup"] & \HH{\K}{\mu} \ar[d, "i"] \\
            & \Q/\Z,
        \end{tikzcd}
    \end{equation*}
    where
    \[
        i = \langle \operatorname{Id}_{\Q/\Z}, - \rangle : \H{2}{\K}{\mu} \Iso \Q/\Z,
    \]
    which concludes as $\langle -{,}- \rangle_M$ is the perfect pairing given by the dualizing module.

    The case $i=0$ then follows by symmetry, as any finite $M$ can be identified with its double dual $M''$.
    We are now left with $i=1$: here it's enough to show that the map
    \[
        \H{1}{\K}{M} \overset{\cup}{\to} \H{1}{\K}{M^\prime}^*
    \]
    induced by the cup product is injective, as then applying the same argument to $M^\prime$ gets us surjectivity, since we are working with finite groups.
    To prove injectivity, we first find a finite $\Gamma$-module $B$ fitting in an exact sequence
    \begin{equation}\label{injSeq}
        0\to M\to B\to C\to 0
    \end{equation}
    such that the induced map
    \[
        \H{1}{\K}{B}\to\H{1}{\K}{B}
    \]
    is zero:
    as usual, we start by injecting $M$ in the induced module $\Ind{\Gamma}{M}$ which is torsion, and by the fundamental property \eqref{indLim} we have

    \[
        0=\H{n}{\K}{\Ind{\Gamma}{M}}=\varinjlim_{\substack{B\subset\Ind{\Gamma}{M} \\B\text{ finite}}}{\H{n}{\K}{B}},
    \]
    which implies the existence of a finite submodule \(B\) with \(\H{n}{\K}{B}=0\). As \(M\) is finite, we can find such a \(B\) which contains \(M\), and then
    the quotient \(C=B/M\) is finite as well.
    Now, we can relate the long exact sequences in cohomology associated to the short exact sequence \eqref{injSeq}
    and its dual using the cup product, which gets us the following commutative diagram (up to a sign, by \eqref{compatCup}) with exact rows:
    \[
        \begin{tikzcd}
            \H{0}{\K}{B} \ar[r] \ar[d, "\cup"] & \H{0}{\K}{C} \ar[r] \ar[d, "\cup"] & \H{1}{\K}{M} \ar[d, "\cup"] \ar[r] & 0 \\
            \H{2}{\K}{B^\prime}^* \ar[r] & \H{2}{\K}{C^\prime}^* \ar[r] & \H{1}{\K}{M^\prime}^*.
        \end{tikzcd}
    \]
    As \(B\) and \(C\) are finite, the first two vertical maps are isomorphisms by the previous cases, and the third one is injective by diagram chasing.

\end{proof}

% Remark: The theorem also holds local fields of positive characteristic and p-torsion-free modules by <ref>

\section{Consequences}

\subsection{Local Class Field Theory}

% As a first application of local duality, we can prove the existence of the Artin map in the local case.

\nocite{*}
\printbibliography

\end{document}
